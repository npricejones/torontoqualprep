\documentclass[a4paper,11pt]{scrartcl}
\usepackage[left=2.3cm,top=2.3cm,bottom=2.3cm,right=2.3cm]{geometry}
\usepackage{natbib}
\usepackage{url}
\usepackage{amsmath, mathtools}
\usepackage[normalem]{ulem}
\usepackage{color}
\usepackage{adjustbox}
\usepackage{float}
\usepackage{chngpage}
\usepackage{authblk}
\usepackage{fancyhdr}
\usepackage{aastex_hack}
\usepackage{wrapfig}

\pagestyle{fancy}

\makeatletter
\let\ps@plain\ps@fancy 
\makeatother

\begin{document}

\section*{Math and General Physics Question 18}

Polarization comes about as a result of oscillations in the electromagnetic field associated with a photon, and by default the polarization is defined as the direction of electric field oscillation. This oscillation is itself an EM wave, variations in the field through space.

Note that all individual photons have a polarization, but we colloquially describe light as polarized when it has a high fraction of photons oscillating in the same direction.

The type of polarization you have depends on the oscillation of the x and y components (relative to the z direction, the direction of propagation) of the electric field and their relative phase. With $E_x$ and $E_y$ perfectly in phase, the polarization can be horizontal ($E_y$=0), vertical ($E_x$=0) or at any relative angle ($E_x\neq0,E_y\neq0$).

\begin{eqnarray}
E_x &=& a_x(t)cos(\omega_0t-\theta_x(t)),\\
E_y &=& a_y(t)cos(\omega_0t-\theta_y(t))\\
\end{eqnarray}

To get a circular polarization, $E_x$ and $E_y$ must be out of phase. For a perfect circle, they are out of phase by $\pm\pi/2$; different phase difference will lead to elliptical polarizations. If $E_y$ is $\pi/2$ behind $E_x$ ($\theta_y-\theta_x = \pi/2$) the direction of oscillation will trace out a circle counterclockwise. If $E_y$ is $\pi/2$ ahead of $E_x$ ($\theta_y-\theta_x = -\pi/2$), the direction of oscillation will trace out a circle clockwise.

\subsection{Stokes Parameters}

The Stokes parameters do not fully describe polarized light but span a range of possible polarizations and are easy to calculate

\begin{itemize}
    \item $I = \langle a_x^2\rangle + \langle a_y^2 \rangle$, intensity
    \item $Q = \langle a_x^2\rangle - \langle a_y^2 \rangle$, polarization along x ($Q>0$) or y ($Q < 0$)
    \item $U = \langle 2a_x a_y \cos(\theta_x-\theta_y)\rangle$ polarization at $\pm \pi/4$
    \item $V = \langle 2a_x a_y \sin(\theta_x-\theta_y)\rangle$,circular polarization
\end{itemize}


\subsection{Cosmological polarization}

In cosmology, polarization is usually measured in terms of E- and B-modes. These modes are named for being curl-free (E), or divergence-free (B). Both are primoridial, but E-modes can be converted to B-modes through gravitational lensing. Since E-modes are much stronger than B-modes, this has made detecting primoridial of B-mdoes difficult, but tantalizing for its possible probing of inflation.

\end{document}